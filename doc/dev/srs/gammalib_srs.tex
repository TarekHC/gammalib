%%%%%%%%%%%%%%%%%%%%%%%%%%%%%%%%%%%%%%%%%%%%%%%%%%%
% GammaLib Software Requirement Specification
%%%%%%%%%%%%%%%%%%%%%%%%%%%%%%%%%%%%%%%%%%%%%%%%%%%

%%%%%%%%%%%%%%%%%%%%%%%%%%%%%%%%%%%%%%%%%%%%%%%%%%%
% Definitions for manual package
%%%%%%%%%%%%%%%%%%%%%%%%%%%%%%%%%%%%%%%%%%%%%%%%%%%
\newcommand{\task}{\mbox{GammaLib}}
\newcommand{\this}{\mbox{\tt \task}}
\newcommand{\shorttype}{\mbox{SRS}}
\newcommand{\doctype}{\mbox{Software Requirement Specification}}
\newcommand{\version}{\mbox{draft}}
\newcommand{\calendar}{\mbox{7 October 2009}}
\newcommand{\auth}{\mbox{J\"urgen Kn\"odlseder}}
\newcommand{\approv}{\mbox{J\"urgen Kn\"odlseder}}


%%%%%%%%%%%%%%%%%%%%%%%%%%%%%%%%%%%%%%%%%%%%%%%%%%%
% Document definition
%%%%%%%%%%%%%%%%%%%%%%%%%%%%%%%%%%%%%%%%%%%%%%%%%%%
\documentclass{article}[12pt,a4]
\usepackage{epsfig}
\usepackage{manual}


%%%%%%%%%%%%%%%%%%%%%%%%%%%%%%%%%%%%%%%%%%%%%%%%%%%
% Begin of document body
%%%%%%%%%%%%%%%%%%%%%%%%%%%%%%%%%%%%%%%%%%%%%%%%%%%
\begin{document}
\frontpage


%%%%%%%%%%%%%%%%%%%%%%%%%%%%%%%%%%%%%%%%%%%%%%%%%%%
% Introduction
%%%%%%%%%%%%%%%%%%%%%%%%%%%%%%%%%%%%%%%%%%%%%%%%%%%
\section{Introduction}

%%%%%%%%%%%%%%%%%%%%%%%%%%%%%%%%%%%%%%%%%%%%%%%%%%%
\subsection{Product Overview}

The \this\ is a vesatile toolbox for the high-level analysis of astronomical gamma-ray data.
It is implemented as a C++ library that is fully scriptable in the Python scripting language.
The library provides core functionalities such as 
data input and output,
interfaces for parameter specifications, and
a reporting and logging interface.
It implements instruments specific functionalities such as
instrument response functions and
data formats.
Instrument specific functionalities share a common interface to allow for extension of the \this\
to include new gamma-ray instruments.
The \this\ provides an abstract data analysis framework that enables simultaneous multi-mission
analysis.

The \this\ is designed as an open source library that is made availably at\break
{\tt http://gammalib.sourceforge.net/}.


%%%%%%%%%%%%%%%%%%%%%%%%%%%%%%%%%%%%%%%%%%%%%%%%%%%
\subsection{Purpose}

The \this\ provides all functionalities required to perform a high-level analysis of astronomical 
gamma-ray data.
High-level analysis means that the data are expected to be calibrated (energy, arrival direction or
detector position) and time tagged.
The \this\ is not designed for preprocessing of data or data calibration purposes.


%%%%%%%%%%%%%%%%%%%%%%%%%%%%%%%%%%%%%%%%%%%%%%%%%%%
\subsection{Scope}

The \this\ is designed for the high-level analysis of astronomical gamma-ray data.
For the purpose of the \this, gamma rays are defined as photons with energies above
few $\sim100$~keV.
The \this\ both applies to data collected by space-based and ground-based instruments.
So far it allows for the analysis of data collected by the following instruments:
CTA.


%%%%%%%%%%%%%%%%%%%%%%%%%%%%%%%%%%%%%%%%%%%%%%%%%%%
\subsection{Reference}

%%%%%%%%%%%%%%%%%%%%%%%%%%%%%%%%%%%%%%%%%%%%%%%%%%%
\subsection{Definition and Abbreviations}

The following abbreviations have been employed in this document:
\begin{itemize}
\item[]{\bf COMPTEL} Compton Telescope (flown aboard CGRO 1991-2000)
\item[]{\bf CTA} Cherenkov Telescope Array
\item[]{\bf Fermi} Fermi Gamma-Ray Space Telescope (launched in 2007)
\item[]{\bf INTEGRAL} International Gamma Ray Laboratory (launched in 2002) 
\item[]{\bf IRAF} 
\item[]{\bf SPI} Spectrometer on INTEGRAL
\end{itemize}


%%%%%%%%%%%%%%%%%%%%%%%%%%%%%%%%%%%%%%%%%%%%%%%%%%%
% Specific Requirements
%%%%%%%%%%%%%%%%%%%%%%%%%%%%%%%%%%%%%%%%%%%%%%%%%%%
\section{Specific Requirements}

%%%%%%%%%%%%%%%%%%%%%%%%%%%%%%%%%%%%%%%%%%%%%%%%%%%
\subsection{External Interface Requirements}

%%%%%%%%%%%%%%%%%%%%%%%%%%%%%%%%%%%%%%%%%%%%%%%%%%%
\subsubsection{User Interfaces}

The \this\ is to designed as C++ API library of which all modules are designed as classes.
A Python interface allows for scripting of library components.

The \this\ supports 2 user interface formats for parameter input:
the IRAF parameter interface (which is also employed for the {\tt ftools} and the 
INTEGRAL and Fermi data analysis frameworks)
and a XML interface (which is also used within the Fermi data analysis framework).

The \this\ provides also a reporting and logging interface that provides the user with 
human-readable information about code execution


%%%%%%%%%%%%%%%%%%%%%%%%%%%%%%%%%%%%%%%%%%%%%%%%%%%
\subsubsection{Hardware Interfaces}

No hardware interfaces are foreseen.


%%%%%%%%%%%%%%%%%%%%%%%%%%%%%%%%%%%%%%%%%%%%%%%%%%%
\subsubsection{Software Interfaces}



%%%%%%%%%%%%%%%%%%%%%%%%%%%%%%%%%%%%%%%%%%%%%%%%%%%
\subsubsection{Communication Protocols}


%%%%%%%%%%%%%%%%%%%%%%%%%%%%%%%%%%%%%%%%%%%%%%%%%%%
\subsubsection{Memory Constraints}


%%%%%%%%%%%%%%%%%%%%%%%%%%%%%%%%%%%%%%%%%%%%%%%%%%%
\subsubsection{Operation}


%%%%%%%%%%%%%%%%%%%%%%%%%%%%%%%%%%%%%%%%%%%%%%%%%%%
\subsubsection{Product function}


%%%%%%%%%%%%%%%%%%%%%%%%%%%%%%%%%%%%%%%%%%%%%%%%%%%
\subsubsection{Assumption and Dependency}

The \this\ should be designed as a standalone package except for a link to the HEASARC
{\tt cfitsio} library.


%%%%%%%%%%%%%%%%%%%%%%%%%%%%%%%%%%%%%%%%%%%%%%%%%%%
\subsection{Software Product Features}


%%%%%%%%%%%%%%%%%%%%%%%%%%%%%%%%%%%%%%%%%%%%%%%%%%%
\subsection{Software System Attributes}

%%%%%%%%%%%%%%%%%%%%%%%%%%%%%%%%%%%%%%%%%%%%%%%%%%%
\subsubsection{Reliability}


%%%%%%%%%%%%%%%%%%%%%%%%%%%%%%%%%%%%%%%%%%%%%%%%%%%
\subsubsection{Availability}


%%%%%%%%%%%%%%%%%%%%%%%%%%%%%%%%%%%%%%%%%%%%%%%%%%%
\subsubsection{Security}


%%%%%%%%%%%%%%%%%%%%%%%%%%%%%%%%%%%%%%%%%%%%%%%%%%%
\subsubsection{Maintainability}


%%%%%%%%%%%%%%%%%%%%%%%%%%%%%%%%%%%%%%%%%%%%%%%%%%%
\subsubsection{Portability}


%%%%%%%%%%%%%%%%%%%%%%%%%%%%%%%%%%%%%%%%%%%%%%%%%%%
\subsubsection{Performance}


%%%%%%%%%%%%%%%%%%%%%%%%%%%%%%%%%%%%%%%%%%%%%%%%%%%
\subsection{Database requirements}


%%%%%%%%%%%%%%%%%%%%%%%%%%%%%%%%%%%%%%%%%%%%%%%%%%%
% Additional Material
%%%%%%%%%%%%%%%%%%%%%%%%%%%%%%%%%%%%%%%%%%%%%%%%%%%
\section{Additional Material}

\end{document} 
