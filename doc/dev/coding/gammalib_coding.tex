%%%%%%%%%%%%%%%%%%%%%%%%%%%%%%%%%%%%%%%%%%%%%%%%%%%
% GammaLib Coding conventions
%%%%%%%%%%%%%%%%%%%%%%%%%%%%%%%%%%%%%%%%%%%%%%%%%%%

%%%%%%%%%%%%%%%%%%%%%%%%%%%%%%%%%%%%%%%%%%%%%%%%%%%
% Definitions for manual package
%%%%%%%%%%%%%%%%%%%%%%%%%%%%%%%%%%%%%%%%%%%%%%%%%%%
\newcommand{\task}{\mbox{GammaLib}}
\newcommand{\this}{\mbox{\tt \task}}
\newcommand{\shorttype}{\mbox{CODING}}
\newcommand{\doctype}{\mbox{Coding Conventions}}
\newcommand{\version}{\mbox{draft}}
\newcommand{\calendar}{\mbox{28 April 2010}}
\newcommand{\auth}{\mbox{J\"urgen Kn\"odlseder}}
\newcommand{\approv}{\mbox{J\"urgen Kn\"odlseder}}


%%%%%%%%%%%%%%%%%%%%%%%%%%%%%%%%%%%%%%%%%%%%%%%%%%%
% Document definition
%%%%%%%%%%%%%%%%%%%%%%%%%%%%%%%%%%%%%%%%%%%%%%%%%%%
\documentclass{article}[12pt,a4]
\usepackage{epsfig}
\usepackage{manual}


%%%%%%%%%%%%%%%%%%%%%%%%%%%%%%%%%%%%%%%%%%%%%%%%%%%
% Begin of document body
%%%%%%%%%%%%%%%%%%%%%%%%%%%%%%%%%%%%%%%%%%%%%%%%%%%
\begin{document}
\frontpage


%%%%%%%%%%%%%%%%%%%%%%%%%%%%%%%%%%%%%%%%%%%%%%%%%%%
% Introduction
%%%%%%%%%%%%%%%%%%%%%%%%%%%%%%%%%%%%%%%%%%%%%%%%%%%
\section{Introduction}

The present document summarises the coding conventions that should be
followed in implementing the \this\ toolbox.
The respect of coherent coding conventions throughout the code
improves code readibility and enhances the portability of the code.


%%%%%%%%%%%%%%%%%%%%%%%%%%%%%%%%%%%%%%%%%%%%%%%%%%%
% System architectural design
%%%%%%%%%%%%%%%%%%%%%%%%%%%%%%%%%%%%%%%%%%%%%%%%%%%
\section{Rules}

The following general rules should be followed:
\begin{itemize}
\item Each function and/or method terminates with a {\tt return;} statement.
\item Put a blank line at the end of each file.
\end{itemize}




%%%%%%%%%%%%%%%%%%%%%%%%%%%%%%%%%%%%%%%%%%%%%%%%%%%
% Documentation
%%%%%%%%%%%%%%%%%%%%%%%%%%%%%%%%%%%%%%%%%%%%%%%%%%%
\section{Documentation}

Code documentation should be done within the source files using Doxygen
compliant annotations.

The following rules apply.


\subsection{File descriptor}

Put a 80 character wide header comment on top of {\bf each file}
(header, source code, templates, etc.).
The header contains
the file name,
a brief description of the file content,
the development period and the name of the person which {\bf initially}
created the file, and
a standard GNU Public License statement.
The header comment is immediately followed by a Doxygen compliant
file description that provides
the file name,
a brief description of the file content, and
the name of the person which {\bf initially} created the file.
\begin{verbatim}
/***************************************************************************
 *                       GMatrix.cpp  -  matrix class                      *
 * ----------------------------------------------------------------------- *
 *  copyright (C) 2006-2010 by Jurgen Knodlseder                           *
 * ----------------------------------------------------------------------- *
 *                                                                         *
 *   This program is free software; you can redistribute it and/or modify  *
 *   it under the terms of the GNU General Public License as published by  *
 *   the Free Software Foundation; either version 2 of the License, or     *
 *   (at your option) any later version.                                   *
 *                                                                         *
 ***************************************************************************/
/**
 * @file GMatrix.cpp
 * @brief GVector class implementation.
 * @author J. Knodlseder
 */
\end{verbatim}

\end{document}

