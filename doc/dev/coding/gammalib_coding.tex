%%%%%%%%%%%%%%%%%%%%%%%%%%%%%%%%%%%%%%%%%%%%%%%%%%%
% GammaLib Coding conventions
%%%%%%%%%%%%%%%%%%%%%%%%%%%%%%%%%%%%%%%%%%%%%%%%%%%

%%%%%%%%%%%%%%%%%%%%%%%%%%%%%%%%%%%%%%%%%%%%%%%%%%%
% Definitions for manual package
%%%%%%%%%%%%%%%%%%%%%%%%%%%%%%%%%%%%%%%%%%%%%%%%%%%
\newcommand{\task}{\mbox{GammaLib}}
\newcommand{\this}{\mbox{\tt \task}}
\newcommand{\shorttype}{\mbox{CCD}}
\newcommand{\doctype}{\mbox{Coding Conventions Document}}
\newcommand{\version}{\mbox{draft}}
\newcommand{\calendar}{\mbox{28 April 2010}}
\newcommand{\auth}{\mbox{J\"urgen Kn\"odlseder}}
\newcommand{\approv}{\mbox{J\"urgen Kn\"odlseder}}


%%%%%%%%%%%%%%%%%%%%%%%%%%%%%%%%%%%%%%%%%%%%%%%%%%%
% Document definition
%%%%%%%%%%%%%%%%%%%%%%%%%%%%%%%%%%%%%%%%%%%%%%%%%%%
\documentclass{article}[12pt,a4]
\usepackage{epsfig}
\usepackage{manual}


%%%%%%%%%%%%%%%%%%%%%%%%%%%%%%%%%%%%%%%%%%%%%%%%%%%
% Begin of document body
%%%%%%%%%%%%%%%%%%%%%%%%%%%%%%%%%%%%%%%%%%%%%%%%%%%
\begin{document}
\frontpage


%%%%%%%%%%%%%%%%%%%%%%%%%%%%%%%%%%%%%%%%%%%%%%%%%%%
% Introduction
%%%%%%%%%%%%%%%%%%%%%%%%%%%%%%%%%%%%%%%%%%%%%%%%%%%
\section{Introduction}

The present document summarises the coding conventions that should be
followed in implementing the \this\ toolbox.
The respect of coherent coding conventions throughout the code
improves code readibility and enhances the portability of the code.


%%%%%%%%%%%%%%%%%%%%%%%%%%%%%%%%%%%%%%%%%%%%%%%%%%%
% General coding rules
%%%%%%%%%%%%%%%%%%%%%%%%%%%%%%%%%%%%%%%%%%%%%%%%%%%
\section{General coding rules}

The following general rules should be followed:
\begin{itemize}
\item[R1] Each function and/or method terminates with a {\tt return;} statement.
\item[R2] Put a blank line at the end of each file.
\item[R3] Use {\tt explicit} for constructors to avoid use of the constructor for unintended.
\item[R4] Do not use tabs (to make code formating independent of editor configurations)
type conversion.
\end{itemize}


%%%%%%%%%%%%%%%%%%%%%%%%%%%%%%%%%%%%%%%%%%%%%%%%%%%
% Coding style
%%%%%%%%%%%%%%%%%%%%%%%%%%%%%%%%%%%%%%%%%%%%%%%%%%%
\section{Coding style}

%%%%%%%%%%%%%%%%%%%%%%%%%%%%%%%%%%%%%%%%%%%%%%%%%%%
\subsection{Code configuration}

The code configuration is controlled via an include file that has to be added on top of
each source and include file.
Each file thus should start with the following code:

\begin{verbatim}
/* __ Includes ___________________________________________________________ */
#ifdef HAVE_CONFIG_H
#include <config.h>
#endif
\end{verbatim}

One of the most common options used throughout \this\ is range checking.
Range checking is particularily important during code development since it allows to catch
memory leaks.
However, range checking is time consuming and thus leads to somewhat slower code.
Range checking can thus be disable during installation of \this\ by using
{\tt ./configure --disable-range-check} during library installation.
Within the code, the following instruction adds range checking that dependens on
the library configuration:

\begin{verbatim}
#if defined(G_RANGE_CHECK)
if (inx < 0 || inx >= m_num)
    throw GException::out_of_range("GVector::operator(int)", inx, m_num);
#endif
\end{verbatim}

Range checking is provided if the {\tt G\_RANGE\_CHECK} macro is defined.


%%%%%%%%%%%%%%%%%%%%%%%%%%%%%%%%%%%%%%%%%%%%%%%%%%%
\subsection{C++ classes}

Memory management should be gathered in a single place to avoid memory leaks.
For this purpose it is best practice to gather the handling of data associated with a C++
class in single places.
For this purpose, each C++ class should have the following {\tt private} methods
for memory management:
\begin{itemize}
\item {\tt init\_members()} initializes all member variables and pointers to well defined
initial values. The class should be fully operational and consistent with these initial
values.
\item {\tt alloc\_members()} allocates all memory required by the class. 
\item {\tt copy\_members(const \&A a)} copies all members from one instance {\tt a}
in the {\tt this} instance.
\item {\tt free\_members()} frees all memory that has been allocated by the class. As
good practice memory pointers should be set to {\tt NULL} if they do not point to
any valid memory. This allows for checking if memory has been allocated before
it is accessed.
\end{itemize}
(in the above notation, {\tt A} is the class name and {\tt a} is an instance of the class.


%%%%%%%%%%%%%%%%%%%%%%%%%%%%%%%%%%%%%%%%%%%%%%%%%%%
% Documentation
%%%%%%%%%%%%%%%%%%%%%%%%%%%%%%%%%%%%%%%%%%%%%%%%%%%
\section{Documentation}

Code documentation should be done within the source files using Doxygen
compliant annotations.

The following rules apply.


\subsection{File descriptor}

Put a 80 character wide header comment on top of {\bf each file}
(header, source code, templates, etc.).
The header contains
the file name,
a brief description of the file content,
the development period and the name of the person which {\bf initially}
created the file, and
a standard GNU Public License statement.
The header comment is immediately followed by a Doxygen compliant
file description that provides
the file name,
a brief description of the file content, and
the name of the person which {\bf initially} created the file.
\begin{verbatim}
/***************************************************************************
 *                       GMatrix.cpp  -  matrix class                      *
 * ----------------------------------------------------------------------- *
 *  copyright (C) 2006-2010 by Jurgen Knodlseder                           *
 * ----------------------------------------------------------------------- *
 *                                                                         *
 *   This program is free software; you can redistribute it and/or modify  *
 *   it under the terms of the GNU General Public License as published by  *
 *   the Free Software Foundation; either version 2 of the License, or     *
 *   (at your option) any later version.                                   *
 *                                                                         *
 ***************************************************************************/
/**
 * @file GMatrix.cpp
 * @brief GVector class implementation.
 * @author J. Knodlseder
 */
\end{verbatim}

\end{document}

