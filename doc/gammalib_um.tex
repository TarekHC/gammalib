%---------------------------------------------------------------------------- 
%                             GammaLib User Manual
%
% Version .........: 0.1pre
% Last modification: 2 November 2006
%----------------------------------------------------------------------------

%%%%%%%%%%%%%%%%%%%%%%%%%%%%%%%%%%%%%%%%%%%%%%%%%%%%%%%%%%%%%%%%%%%%%%%%%%%%%
% Definitions for manual package
%%%%%%%%%%%%%%%%%%%%%%%%%%%%%%%%%%%%%%%%%%%%%%%%%%%%%%%%%%%%%%%%%%%%%%%%%%%%%
\newcommand{\task}{\mbox{GammaLib}}
\newcommand{\this}{\mbox{\tt \task}}
\newcommand{\version}{\mbox{0.1pre}}
\newcommand{\calendar}{\mbox{2 November 2006}}


%%%%%%%%%%%%%%%%%%%%%%%%%%%%%%%%%%%%%%%%%%%%%%%%%%%%%%%%%%%%%%%%%%%%%%%%%%%%%
% Document definition
%%%%%%%%%%%%%%%%%%%%%%%%%%%%%%%%%%%%%%%%%%%%%%%%%%%%%%%%%%%%%%%%%%%%%%%%%%%%%
\documentclass{article}[12pt,a4]
\usepackage{epsfig}
\usepackage{manual}


%%%%%%%%%%%%%%%%%%%%%%%%%%%%%%%%%%%%%%%%%%%%%%%%%%%%%%%%%%%%%%%%%%%%%%%%%%%%%
% Begin of document body
%%%%%%%%%%%%%%%%%%%%%%%%%%%%%%%%%%%%%%%%%%%%%%%%%%%%%%%%%%%%%%%%%%%%%%%%%%%%%
\begin{document}
\frontpage


%%%%%%%%%%%%%%%%%%%%%%%%%%%%%%%%%%%%%%%%%%%%%%%%%%%%%%%%%%%%%%%%%%%%%%%%%%%%%
% Introduction
%%%%%%%%%%%%%%%%%%%%%%%%%%%%%%%%%%%%%%%%%%%%%%%%%%%%%%%%%%%%%%%%%%%%%%%%%%%%%
\section{Introduction}

%%%%%%%%%%%%%%%%%%%%%%%%%%%%%%%%%%%%%%%%%%%%%%%%%%%%%%%%%%%%%%%%%%%%%%%%%%%%%
\subsection{Motivation}

Again another library for scientific calculus?
Yes and no.
\this\ indeed provides a number of functionnalities that are also 
found in other libraries, but \this\ goes beyond these libraries in 
that it provides tools that are rather specific to the analysis of 
high-energy astronomy data.
The idea was to put everything together in one single place, limiting 
thus the dependencies.

It all started when I was preparing the software 
for the scientific exploitation of the data from the SPI telescope 
aboard INTEGRAL.
Instead of writing a number of indepedent executables, I decided to 
write a set of powerful C++ classes that provide the required 
functionalities, such as data handling, response calculation, image 
convolution, model fitting, etc.
Analysis executables were then simple clients of these C++ classes and 
basically only provided the user interface.
The C++ classes were designed quite general, so that it was obvious to 
attempt their utilisation also for the analysis of other data.
However, it turned out that the SPI classes were still too spezialised 
for this purpose, and a new, more general library was required to 
fullfil my goal.

I started the work when I needed sparse matrix capabilities for my 
SPI model fitting routines.
With the ongoing INTEGRAL mission, the datasets became increasinly 
larger and the number of fit parameters required to model these data 
increased accordingly.
The initial \this\ thus had vector and matrix classes which it 
provided to the SPI tools library.


%%%%%%%%%%%%%%%%%%%%%%%%%%%%%%%%%%%%%%%%%%%%%%%%%%%%%%%%%%%%%%%%%%%%%%%%%%%%%
\subsection{Using \this}

TBD: How to install, how to write code ..


%%%%%%%%%%%%%%%%%%%%%%%%%%%%%%%%%%%%%%%%%%%%%%%%%%%%%%%%%%%%%%%%%%%%%%%%%%%%%
% GammaLib objects
%%%%%%%%%%%%%%%%%%%%%%%%%%%%%%%%%%%%%%%%%%%%%%%%%%%%%%%%%%%%%%%%%%%%%%%%%%%%%
\section{\this\ objects}

%%%%%%%%%%%%%%%%%%%%%%%%%%%%%%%%%%%%%%%%%%%%%%%%%%%%%%%%%%%%%%%%%%%%%%%%%%%%%
\subsection{Vectors}

%----------------------------------------------------------------------------
\subsubsection{General}

A vector is a one-dimensional array of successive {\tt double} type values.
Vectors are handled in \this\ by {\tt GVector} objects.
On construction, the dimension of the vector has to be specified.
In other words
\begin{verbatim}
 GVector vector;                        // WRONG: constructor needs dimension
\end{verbatim}
is not allowed.
The minimum dimension of a vector is 1, i.e. there is no such thing 
like an empty vector:
\begin{verbatim}
 GVector vector(0);                     // WRONG: empty vector not allowed
\end{verbatim}
The correct allocation of a vector is done using
\begin{verbatim}
 GVector vector(10);                    // Allocates a vector with 10 elements
\end{verbatim}
On allocation, all elements of a vector are set to 0.
Vectors may also be allocated by copying from another vector
\begin{verbatim}
 GVector vector(10);                    // Allocates a vector with 10 elements
 GVector another = vector;              // Allocates another vector with 10 elements
\end{verbatim}
or by using 
\begin{verbatim}
 GVector vector = GVector(10);          // Allocates a vector with 10 elements
\end{verbatim}

Vector elements are accessed using the {\tt ( )} operator:
\begin{verbatim}
 GVector vector(10);                    // Allocates a vector with 10 elements
 for (int i = 0; i < 10; ++i)
   vector(i) = (i+1)*10.0;              // Set elements 10, 20, ..., 100
 for (int i = 0; i < 10; ++i)
   cout << vector(i) << endl;           // Dump all elements, one by row 
\end{verbatim}
The content of a vector may also be dumped using
\begin{verbatim}
 cout << vector << endl;                // Dump entire vector
\end{verbatim}
which in the above example will put the sequence
\begin{verbatim}
 10 20 30 40 50 60 70 80 90 100
\end{verbatim}
on the screen.


%----------------------------------------------------------------------------
\subsubsection{Vector arithmetics}

Vectors can be very much handled like {\tt double} type variables
with the difference that operations are performed on each element of 
the vector.
The complete list of fundamental vector operators is:
\begin{verbatim}
 c = a + b;                             // Vector + Vector addition
 c = a + s;                             // Vector + Scalar addition
 c = s + b;                             // Scalar + Vector addition
 c = a - b;                             // Vector - Vector subtraction
 c = a - s;                             // Vector - Scalar subtraction
 c = s - b;                             // Scalar - Vector subtraction
 s = a * b;                             // Vector * Vector multiplication (dot product)
 c = a * s;                             // Vector * Scalar multiplication
 c = s * b;                             // Scalar * Vector multiplication
 c = a / s;                             // Vector * Scalar division
\end{verbatim}
where {\tt a}, {\tt b} and {\tt c} are of type {\tt GVector} and 
{\tt s} is of type {\tt double}.
Note in particular the combination of {\tt GVector} and {\tt double} 
type objects in addition, subtraction, multiplication and division.
In these cases the specified operation is applied to each of the 
vector elements.
It is also obvious that only vector of identicial dimension can occur 
in vector operations.
Dimension errors can be catched by the {\tt try} - {\tt catch} 
functionality:
\begin{verbatim}
 try {
   GVector a(10);                       
   GVector b(11);
   GVector c = a + b;                   // WRONG: Vectors have incompatible dimensions
 }
 catch (GVector::vector_mismatch &e) {
   cout << e.what() << endl;            // Dimension exception is catched here
   throw;
 }
\end{verbatim}
Further vector operations are
\begin{verbatim}
 c = a;                                 // Vector assignment
 c = s;                                 // Scalar assignment
 s = c(index);                          // Vector element access
 c += a;                                // c = c + a;
 c -= a;                                // c = c - a;
 c += s;                                // c = c + s;
 c -= s;                                // c = c - s;
 c *= s;                                // c = c * s;
 c /= s;                                // c = c / s;
 c = -a;                                // Vector negation
\end{verbatim}
Finally, the comparison operators
\begin{verbatim}
 int equal   = (a == b);                // True if all elements equal
 int unequal = (a != b);                // True if at least one elements unequal
\end{verbatim}
allow to compare all elements of a vector. If all elements are
identical, the {\tt ==} operator returns true, otherwise false.
If at least one element differs, the {\tt !=} operator returns true, 
is all elements are identical it returns false.

In addition to the operators, the following mathematical functions 
can be applied to vectors:
\begin{verbatim}
    acos         atan         exp          sin          tanh
    acosh        atanh        fabs         sinh
    asin         cos          log          sqrt
    asinh        cosh         log10        tan
\end{verbatim}
Again, these functions should be understood to be applied element wise.
They all take a vector as argument and produce a vector as result.
For example
\begin{verbatim}
 c = sin(a);
\end{verbatim}
attributes the sine of each element of vector {\tt a} to vector 
{\tt c}.
Additional implemented functions are
\begin{verbatim}
 c = cross(a, b);                       // Vector cross product (for 3d only)
 s = norm(a);                           // Vector norm |a|
 s = min(a);                            // Minimum element of vector
 s = max(a);                            // Maximum element of vector
 s = sum(a);                            // Sum of vector elements
\end{verbatim}
Finally, a small number of vector methods have been implemented:
\begin{verbatim}
 int n = a.size();                      // Returns dimension of vector
 int n = a.non_zeros();                 // Returns number of non-zero elements in vector
\end{verbatim}


%%%%%%%%%%%%%%%%%%%%%%%%%%%%%%%%%%%%%%%%%%%%%%%%%%%%%%%%%%%%%%%%%%%%%%%%%%%%%
\subsection{Matrixes}

%----------------------------------------------------------------------------
\subsubsection{General}

A matrix is a two-dimensional array of {\tt double} type values, 
arranged in rows and columns.
Matrixes are handled in \this\ by {\tt GMatrix} objects and the 
derived classes {\tt GSymMatrix} and {\tt GSparseMatrix} (see section 
\ref{sec:matrix:storage}).
On construction, the dimension of the matrix has to be specified:
\begin{verbatim}
 GMatrix matrix(10,20);                 // Allocates 10 rows and 20 columns
\end{verbatim}
Similar to vectors, there is no such thing as a matrix without 
dimensions in \this.


%----------------------------------------------------------------------------
\subsubsection{Matrix storage classes}
\label{sec:matrix:storage}

In the most general case, the {\tt rows} and {\tt columns} of a 
matrix are stored in a continuous array of 
${\tt rows} \times {\tt columns}$
memory locations.
This storage type is referred to as a {\em full matrix}, and is 
implemented by the class {\tt GMatrix}.
Operations on full matrixes are in general relatively fast, but 
memory requirements may be important to hold all the elements.
In general matrixes are stored by \this\ column-wise (or in 
column-major format).
For example, the matrix
\begin{verbatim}
    1  2  3  4  5
    6  7  8  9 10
   11 12 13 14 15 
\end{verbatim}
is stored in memory as
\begin{verbatim}
    |  1  6 11 |  2  7 12 |  3  8 13 |  4  9 14 |  5 10 15 |
\end{verbatim}

Many physical or mathematical problems treat with a subclass of 
matrixes that is symmetric, i.e. for which the element
{\tt (row,col)} is identical to the element {\tt (col,row)}.
In this case, the duplicated elements need not to be stored.
The derived class {\tt GSymMatrix} implements such a storage type.
{\tt GSymMatrix} stores the lower-left triangle of the matrix in 
column-major format.
For illustration, the matrix
\begin{verbatim}
    1  2  3  4
    2  5  6  7
    3  6  8  9
    4  7  9 10
\end{verbatim}
is stored in memory as
\begin{verbatim}
    |  1  2  3  4 |  5  6  7 |  8  9 | 10 |
\end{verbatim}
This divides the storage requirements to hold the matrix elements by 
almost a factor of two.

Finally, quite often one has to deal with matrixes that contain a 
large number of zeros.
Such matrixes are called {\em sparse matrixes}.
If only the non-zero elements of a sparse matrix are stored the memory 
requirements are considerably reduced.
This goes however at the expense of matrix element access, which has 
become now more complex.
In particular, filling efficiently a sparse matrix is a non-trivial 
problem (see section \ref{sec:matrix:filling}).
Sparse matrix storage is implemented in \this\ by the derived class 
{\tt GSparseMatrix}.
A {\tt GSparseMatrix} object contains three one-dimensional arrays to 
store the matrix elements: 
a {\tt double} type array that contains in continuous column-major order 
all non-zero elements, 
an {\tt int} type array that contains for each non-zero element the 
row number of its location, and
an {\tt int} type array that contains the storage location of the 
first non-zero element for each matrix column.
To illustrate this storage format, the matrix
\begin{verbatim}
    1  0  0  7
    2  5  0  0
    3  0  6  0
    4  0  0  8
\end{verbatim}
is stored in memory as 
\begin{verbatim}
    |  1  2  3  4 |  5 |  6 |  7  8 |  Matrix elements
    |  0  1  2  3 |  1 |  2 |  0  3 |  Row indices for all elements
    |  0          |  4 |  5 |  6    |  Storage location of first element of each column
\end{verbatim}
This example is of course not very economic, since the total number of 
Bytes used to store the matrix is 
$8 \times 8 + (8 + 4) \times 4 = 112$ Bytes, while a full $4 \times 4$
matrix is stored in
$(4 \times 4) \times 8 = 128$ Bytes (recall: a {\tt double} type 
values takes 8 Bytes, an {\tt int} type value takes 4 Bytes).
For realistic large systems, however, the gain in memory space can be 
dramatical.

The usage of the {\tt GMatrix}, {\tt GSymMatrix} and {\tt GSparseMatrix} 
classes is analoguous in that they implement basically all functions 
and methods in an identical way.
So from the semantics the user has not to worry about the storage 
class. 
However, matrix element access speeds are not identical for all 
storage types, and if performance is an issue (as it certainly always 
will be), the user has to consider matrix access more carefully
(see section \ref{sec:matrix:filling}).

Matrix allocation is performed using the constructors:
\begin{verbatim}
 GMatrix       A(10,20);                // Full 10 x 20 matrix
 GSymMatrix    B(10,10);                // Symmetric 10 x 10 matrix
 GSparseMatrix C(1000,10000);           // Sparse 1000 x 10000 matrix

 GMatrix       A(0,0);                  // WRONG: empty matrix not allowed
 GSymMatrix    B(20,22);                // WRONG: symmetric matrix requested
\end{verbatim}
In the constructor, the first argument specifies the number of rows, 
the second the number of columns: {\tt A(row,column)}.
A symmetric matrix needs of course an equal number of rows and columns.
And an empty matrix is not allowed.
All matrix elements are initialised to 0 by the matrix allocation.

Matrix elements are accessed by the {\tt A(row,col)} function,
where {\tt row} and {\tt col} start from 0 for the first row or column 
and run up to the number of rows or columns minus 1:
\begin{verbatim}
 for (int row = 0; row < n_rows; ++row) {
   for (int col = 0; col < n_cols; ++col)
     A(row,col) = (row+col)/2.0;        // Set value of matrix element
 }
 ...
 double sum2 = 0.0;
 for (int row = 0; row < n_rows; ++row) {
   for (int col = 0; col < n_cols; ++col)
     sum2 *= A(row,col) * A(row,col);   // Get value of matrix element
 }
\end{verbatim}

The content of a matrix can be visualised using
\begin{verbatim}
 cout << A << endl;                     // Dump matrix
\end{verbatim}


%----------------------------------------------------------------------------
\subsubsection{Matrix arithmetics}

The following description of matrix arithmetics applies to all storage 
classes (see section \ref{sec:matrix:storage}).
The following matrix operators have been implemented in \this:
\begin{verbatim}
 C = A + B;                             // Matrix Matrix addition
 C = A - B;                             // Matrix Matrix subtraction
 C = A * B;                             // Matrix Matrix multiplication
 C = A * v;                             // Matrix Vector multiplication
 C = A * s;                             // Matrix Scalar multiplication
 C = s * A;                             // Scalar Matrix multiplication
 C = A / s;                             // Matrix Scalar division
 C = -A;                                // Negation
 A += B;                                // Matrix inplace addition
 A -= B;                                // Matrix inplace subtraction
 A *= B;                                // Matrix inplace multiplications
 A *= s;                                // Matrix inplace scalar multiplication
 A /= s;                                // Matrix inplace scalar division
\end{verbatim}
The comparison operators
\begin{verbatim}
 int equal   = (A == B);                // True if all elements equal
 int unequal = (A != B);                // True if at least one elements unequal
\end{verbatim}
allow to compare all elements of a matrix. 
If all elements are identical, the {\tt ==} operator returns true, 
otherwise false.
If at least one element differs, the {\tt !=} operator returns true, 
is all elements are identical it returns false.


%----------------------------------------------------------------------------
\subsubsection{Matrix methods and functions}

A number of methods has been implemented to manipulate matrixes.
The method
\begin{verbatim}
 A.clear();                             // Set all elements to 0
\end{verbatim}
sets all elements to 0.
The methods
\begin{verbatim}
 int rows = A.rows();                   // Returns number of rows in matrix
 int cols = A.cols();                   // Returns number of columns in matrix
\end{verbatim}
provide access to the matrix dimensions, the methods
\begin{verbatim}
 double sum = A.sum();                  // Sum of all elements in matrix
 double min = A.min();                  // Returns minimum element of matrix
 double max = A.max();                  // Returns maximum element of matrix
\end{verbatim}
inform about some matrix properties.
The methods
\begin{verbatim}
 GVector v_row    = A.extract_row(row); // Puts row in vector
 GVector v_column = A.extract_col(col); // Puts column in vector
\end{verbatim}
extract entire rows and columns from a matrix.
Extraction of lower or upper triangle parts of a matrix into another
is performed using
\begin{verbatim}
 B = A.extract_lower_triangle();        // B holds lower triangle
 B = A.extract_upper_triangle();        // B holds upper triangle
\end{verbatim}
{\tt B} is of the same storage class as {\tt A}, except for the case 
that {\tt A} is a {\tt GSymMatrix} object. 
In this case, {\tt B} will be a full matrix of type {\tt GMatrix}.

The methods
\begin{verbatim}
 A.insert_col(v_col,col);               // Puts vector in column
 A.add_col(v_col,col);                  // Add vector to column
\end{verbatim}
inserts or adds the elements of a vector into a matrix column.
Note that no row insertion routines have been implemented (so far) 
since they would be less efficient (recall that all matrix types are 
stored in column-major format).

Conversion from one storage type to another is performed using
\begin{verbatim}
 B = A.convert_to_full();               // Converts A -> GMatrix
 B = A.convert_to_sym();                // Converts A -> GSymMatrix
 B = A.convert_to_sparse();             // Converts A -> GSparseMatrix
\end{verbatim}
Note that {\tt convert\_to\_sym()} can only be applied to a matrix that 
is indeed symmetric.

The transpose of a matrix can be obtained by using one of
\begin{verbatim}
 A.transpose();                         // Transpose method
 B = transpose(A);                      // Transpose function
\end{verbatim}

The absolute value of a matrix is provided by
\begin{verbatim}
 B = fabs(A);                           // B = |A|
\end{verbatim}


%----------------------------------------------------------------------------
\subsubsection{Matrix factorisations}

A general tool of numeric matrix calculs is factorisation.

Solve linear equation {\tt Ax = b}.
Inverse a matrix (by solving successively {\tt Ax = e}, where {\tt e} 
are the unit vectors for all dimensions).

For symmetric and positive definite matrices the most efficient 
factorisation is the Cholesky decomposition.
The following code fragment illustrates the usage:
\begin{verbatim}
 GMatrix A(n_rows, n_cols};
 GVector x(n_rows};
 GVector b(n_rows};
 ...
 A.cholesky_decompose();                // Perform Cholesky factorisation
 x = A.cholesky_solver(b);              // Solve Ax=b for x
\end{verbatim}
Note that once the function {\tt A.cholesky\_decompose()} has been 
applied, the original matrix content has been replaced by its Cholesky 
decomposition.
Since the Cholesky decomposition can be performed inplace (i.e. 
without the allocation of additional memory to hold the result), the 
matrix replacement is most memory economic.
In case that the original matrix should be kept, one may either copy 
it before into another {\tt GMatrix} object or use the function
\begin{verbatim}
 GMatrix L = cholesky_decompose(A);
 x = L.cholesky_solver(b);
\end{verbatim}

A symmetric and positif definite matrix can be inverted using the 
Cholesky decomposition using
\begin{verbatim}
 A.cholesky_invert();                   // Inverse matrix using Cholesky fact.
\end{verbatim}
Alternatively, the function 
\begin{verbatim}
 GMatrix A_inv = cholesky_invert(A);
\end{verbatim}
may be used.

The Cholesky decomposition, solver and inversion
routines may also be applied to matrices that contain rows or
columns that are filled by zeros.
In this case the functions provide the option to (logically)
compress the matrices by skipping the zero rows and columns during
the calculation.

For compressed matrix Cholesky factorisation, only the non-zero rows 
and columns have to be symmetric and positive definite.
In particular, the full matrix may even be non-symmetric.


%----------------------------------------------------------------------------
\subsubsection{Sparse matrixes}

The only exception that does not work is
\begin{verbatim}
 GSparseMatrix A(10,10);
 A(0,0) = A(1,1) = A(2,2) = 1.0;        // WRONG: Cannot assign multiple at once
\end{verbatim}
In this case the value {\tt 1.0} is only assigned to the last 
element, i.e. {\tt A(2,2)}, the other elements will remain
{\tt 0}.
This feature has to do with the way how the compiler translates
the code and how \this\ implements sparse matrix filling.
{\tt GSparseMatrix} provides a pointer for a new element to be 
filled.
Since there is only one such {\em fill pointer}, only one element can 
be filled at once in a statement.
{\bf So it is strongly advised to avoid multiple matrix element 
assignment in a single row.}
Better write the above code like
\begin{verbatim}
 GSparseMatrix A;
 A(0,0) = 1.0;
 A(1,1) = 1.0;
 A(2,2) = 1.0;
\end{verbatim}
This way, element assignment works fine.

Inverting a sparse matrix produces in general a full matrix, so the 
inversion function should be used with caution.
Note that a full matrix that is stored in sparse format takes roughly
twice the memory than a normal {\tt GMatrix} object.
If nevertheless the inverse of a sparse matrix should be examined, it 
is recommended to perform the analysis column-wise:
\begin{verbatim}
 GSparseMatrix A(rows,cols);            // Allocate sparse matrix
 GVector       unit(rows);              // Allocate vector
 ...
 A.cholesky_decompose();                // Factorise matrix

 // Column-wise solving the matrix equation
 for (int col = 0; col < cols; ++col) {
   unit(col) = 1.0;                     // Set unit vector
   GVector x = cholesky_solver(unit);   // Get column x of inverse
   ...
   unit(col) = 0.0;                     // Clear unit vector for next round
 }
\end{verbatim}


%----------------------------------------------------------------------------
\subsubsection{Filling sparse matrixes}
\label{sec:matrix:filling}

The filling of a sparse matrix is a tricky issue since the storage
of the elements depends on their distribution in the matrix.
If one would know beforehand this distribution, sparse matrix filling
would be easy and fast.
In general, however, the distribution is not known a priori, and
matrix filling may become a quite time consuming task.

If a matrix has to be filled element by element, the access through
the operator
\begin{verbatim}
 m(row,col) = value;
\end{verbatim}
may be mandatory.
In principle, if a new element is inserted into a matrix a new memory
cell has to be allocated for this element, and other elements may be 
moved.
Memory allocation is quite time consuming, and to reduce the overhead,
{\tt GSparseMatrix} can be configured to allocate memory in bunches.
By default, each time more matrix memory is needed, {\tt GSparseMatrix}
allocates 512 cells at once (or 6144 Bytes since each element requires 
a {\tt double} and a {\tt int} storage location).
If this amount of memory is not adequat one may change this value by
using
\begin{verbatim}
 m.set_mem_block(size);
\end{verbatim}
where {\tt size} is the number of matrix elements that should be
allocated at once (corresponding to a total memory of 
$12 \times {\tt size}$ Bytes).

Alternatively, a matrix may be filled column-wise using the functions
\begin{verbatim}
 m.insert_col(vector,col);              // Insert a vector in column
 m.add_col(vector,col);                 // Add content of a vector to column
\end{verbatim}
While {\tt insert\_col} sets the values of column {\tt col} (deleting 
thus any previously existing entries), {\tt add\_col} adds the content 
of {\tt vector} to all elements of column {\tt col}.
Using these functions is considerably more rapid than filling individual 
values.

Still, if the matrix is big (i.e. severeal thousands of rows and 
columns), filling individual columns may still be slow.
To speed-up dynamical matrix filling, an internal fill-stack has been 
implemented in {\tt GSparseMatrix}.
Instead of inserting values column-by-column, the columns are stored 
in a stack and filled into the matrix once the stack is full.
This reduces the number of dynamic memory allocations to let the 
matrix grow as it is built.
By default, the internal stack is disabled.
The stack can be enabled and used as follows:
\begin{verbatim}
 m.stack_init(size, entries);           // Initialise stack
 ...
 m.add_col(vector,col);                 // Add columns
 ...
 m.stack_destroy();                     // Flush and destory stack
\end{verbatim}
The method {\tt stack\_init} initialises a stack with a number of 
{\tt size} elements and a maximum of {\tt entries} columns.
The larger the values {\tt size} and {\tt entries} are chosen, the
more efficient the stack works.
The total amount of memory of the stack can be estimated as
$12 \times {\tt size} + 8 \times {\tt entries}$ Bytes.
If a rough estimate of the total number of non-zero elements is
available it is recommended to set {\tt size} to this value.
As a rule of thumb, {\tt size} should be at least of the dimension of 
either the number of rows or the number of columns of the matrix 
(take the maximum of both).
{\tt entries} is best set to the number of columns of the matrix.
If memory limits are an issue smaller values may be set, but if the
values are too small, the speed increase may become negligible (or 
stack-filling may even become slower than normal filling).

Stack-filling only works with the method {\tt add\_col}.
Note also that filling sub-sequently the same column leads to stack 
flushing.
In the code
\begin{verbatim}
 for (int col = 0; col < 100; ++col) {
   column      = 0.0;                   // Reset column
   column(col) = col;                   // Set column
   m.add_col(column,col);               // Add column
 }   
\end{verbatim}
stack flushing occurs in each loop, and consequently, the 
stack-filling approach will be not very efficient (it 
would probably be even slover than normal filling).
If successive operations are to be performed on columns, 
it is better to perform them before adding.
The code
\begin{verbatim}
 column = 0.0;                          // Reset column
 for (int col = 0; col < 100; ++col)
   column(col) = col;                   // Set column
 m.add_col(column,col);                 // Add column
\end{verbatim}
would be far more efficient.

A avoidable overhead occurs for the case that the column to be added 
is sparse.
The vector may contain many zeros, and {\tt GSparseMatrix} has to 
filter them out.
If the sparsity of the column is known, this overhead can be avoided 
by directly passing a compressed array to {\tt add\_col}:
\begin{verbatim}
 int     number = 5;                    // 5 elements in array
 double* values = new double[number];   // Allocate values
 int*    rows   = new int[number];      // Allocate row index
 ...
 m.stack_init(size, entries);           // Initialise stack
 ...
 for (int i = 0; i < number; ++i) {     // Initialise array
   values[i] = ...                      // ... set values
   rows[i]   = ...                      // ... set row indices
 }
 ...
 m.add_col(values,rows,number,col);     // Add array
 ...
 m.stack_destroy();                     // Flush and destory stack
 ...
 delete [] values;                      // Free array
 delete [] rows;
\end{verbatim}

The method {\tt add\_col} calls the method {\tt stack\_push\_column}
for stack filling.
{\tt add\_col} is more general than {\tt stack\_push\_column} in that 
it decides which of stack- or direct filling is more adequate.
In particular, {\tt stack\_push\_column} may refuse pushing a column 
onto the stack if there is not enough space.
In that case, {\tt stack\_push\_column} returns a non-zero value that 
corresponds to the number of non-zero elements in the vector that 
should be added.
However, it is recommended to not use {\tt stack\_push\_column} and 
call instead {\tt add\_col}.

The method {\tt stack\_destroy} is used to flush and destroy the
stack. 
After this call the stack memory is liberated.
If the stack should be flushed without destroying it, the method
{\tt stack\_flush} may be used:
\begin{verbatim}
 m.stack_init(size, entries);           // Initialise stack
 ...
 m.add_col(vector,col);                 // Add columns
 ...
 m.stack_flush();                       // Simply flush stack
\end{verbatim}
Once flushed, the stack can be filled anew.

Note that stack flushing is not automatic!
This means, if one trys to use a matrix for calculs without flushing, 
the calculs may be wrong.
{\bf If a stack is used for filling, always flush the stack before using 
the matrix.}


%%%%%%%%%%%%%%%%%%%%%%%%%%%%%%%%%%%%%%%%%%%%%%%%%%%%%%%%%%%%%%%%%%%%%%%%%%%%%
% Code reference
%%%%%%%%%%%%%%%%%%%%%%%%%%%%%%%%%%%%%%%%%%%%%%%%%%%%%%%%%%%%%%%%%%%%%%%%%%%%%
\clearpage
\section{Code reference}

%%%%%%%%%%%%%%%%%%%%%%%%%%%%%%%%%%%%%%%%%%%%%%%%%%%%%%%%%%%%%%%%%%%%%%%%%%%%%
\subsection{{\tt GVector}}

%%%%%%%%%%%%%%%%%%%%%%%%%%%%%%%%%%%%%%%%%%%%%%%%%%%%%%%%%%%%%%%%%%%%%%%%%%%%%
\subsection{{\tt GMatrix}}

\end{document} 
