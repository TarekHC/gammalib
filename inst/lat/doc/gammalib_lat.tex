%%%%%%%%%%%%%%%%%%%%%%%%%%%%%%%%%%%%%%%%%%%%%%%%%%%
% GammaLib LAT Specific Interface
%%%%%%%%%%%%%%%%%%%%%%%%%%%%%%%%%%%%%%%%%%%%%%%%%%%

%%%%%%%%%%%%%%%%%%%%%%%%%%%%%%%%%%%%%%%%%%%%%%%%%%%
% Definitions for manual package
%%%%%%%%%%%%%%%%%%%%%%%%%%%%%%%%%%%%%%%%%%%%%%%%%%%
\newcommand{\task}{\mbox{GammaLib}}
\newcommand{\this}{\mbox{\tt \task}}
\newcommand{\shorttype}{\mbox{LAT}}
\newcommand{\doctype}{\mbox{LAT specific interface}}
\newcommand{\version}{\mbox{draft}}
\newcommand{\calendar}{\mbox{12 December 2010}}
\newcommand{\auth}{\mbox{J\"urgen Kn\"odlseder}}
\newcommand{\approv}{\mbox{J\"urgen Kn\"odlseder}}


%%%%%%%%%%%%%%%%%%%%%%%%%%%%%%%%%%%%%%%%%%%%%%%%%%%
% Document definition
%%%%%%%%%%%%%%%%%%%%%%%%%%%%%%%%%%%%%%%%%%%%%%%%%%%
\documentclass{article}[12pt,a4]
\usepackage{epsfig}
\usepackage{manual}


%%%%%%%%%%%%%%%%%%%%%%%%%%%%%%%%%%%%%%%%%%%%%%%%%%%
% Begin of document body
%%%%%%%%%%%%%%%%%%%%%%%%%%%%%%%%%%%%%%%%%%%%%%%%%%%
\begin{document}
\frontpage


%%%%%%%%%%%%%%%%%%%%%%%%%%%%%%%%%%%%%%%%%%%%%%%%%%%
% Introduction
%%%%%%%%%%%%%%%%%%%%%%%%%%%%%%%%%%%%%%%%%%%%%%%%%%%
\section{Introduction}



%%%%%%%%%%%%%%%%%%%%%%%%%%%%%%%%%%%%%%%%%%%%%%%%%%%
% Science Tools
%%%%%%%%%%%%%%%%%%%%%%%%%%%%%%%%%%%%%%%%%%%%%%%%%%%
\section{Science Tools}

%%%%%%%%%%%%%%%%%%%%%%%%%%%%%%%%%%%%%%%%%%%%%%%%%%%
\subsection{Binned analysis}

The binned likelihood analysis is implemented by {\tt BinnedLikelihood.cxx}.
Evaluation of the binned likelihood is done by the method
{\tt BinnedLikelihood::value}
using
\begin{equation}
\ln L = \sum_{e_i > 0} n_i \ln e_i - N_{\rm pred}
\end{equation}
where
$n_i$ is the number of counts in bin $i$,
$e_i$ is the model value for bin $i$,
and
$N_{\rm pred}$ is the total number of predicted counts.

$e_i$ is computed by
{\tt BinnedLikelihood::computeModelMap}
by looping over all source models $m$ and by summing up the predicted number of
counts for all pixels $i$:
\begin{equation}
e_i = \sum_m e_{i,m} .
\end{equation}
For a given bin and model, $e_{i,m}$ is computed by 
{\tt BinnedLikelihood::pixelCounts} which implements the Trapezoid rule
$\int_a^b f(x) dx \approx (b-a) \frac{f(a) + f(b)}{2}$
to compute the predicted number of counts in a pixel by integrating the
model prediction over the spectrum:
\begin{equation}
e_{i,m} = \frac{
I(E_i^{\rm min}) s_{i,m}(E_i^{\rm min}) + I(E_i^{\rm max}) s_{i,m}(E_i^{\rm max})}{2} 
( E_i^{\rm max} - E_i^{\rm min} )
\end{equation}
where
$I(E)$ is the spectral function,
$s_{i,m}(E_i^{\rm min})$ and $s_{i,m}(E_i^{\rm max})$ are the values of bin $i$ for
source model $m$ at the minimum and maximum energy boundaries 
$E_i^{\rm min}$ and $E_i^{\rm max}$ of the bin.

$N_{\rm pred}$ is computed by
{\tt BinnedLikelihood::computeModelMap}
by looping over all source models and by summing up the predicted number of
counts for a given model:
\begin{equation}
N_{\rm pred} = \sum_m N_{\rm pred}^{(m)} .
\end{equation}
The predicted number of counts for a given model is computed by
{\tt BinnedLikelihood::NpredValue}
which sums the prediction over all energy bins $E_k$:
\begin{equation}
N_{\rm pred}^{(m)} = \sum_k N_{\rm pred}^{(m)}(E_k) .
\end{equation}
For a given energy bin, $N_{\rm pred}^{(m)}(E_k)$ is computed by 
{\tt Source::pixelCounts} which implements the Trapezoid rule
$\int_a^b f(x) dx \approx (b-a) \frac{f(a) + f(b)}{2}$
to compute the predicted number of counts in a pixel by integrating the
model prediction over the spectrum:
\begin{equation}
N_{\rm pred}^{(m)}(E_k) = \frac{
I(E_{\rm min}) S^{(m)}(E_k^{\rm min}) + I(E_{\rm max}) S^{(m)}(E_k^{\rm max})}{2} 
( E_k^{\rm max} - E_k^{\rm min} )
\end{equation}
where
$I(E)$ is the spectral function,
$S^{(m)}(E_k^{\rm min})$ and $S^{(m)}(E_k^{\rm max})$ are the spatially integrated source
model $i$ at the energies $E_k^{\rm min}$ and $E_k^{\rm max}$.
$S^{(m)}(E_k)$ is computed by
{\tt SourceMap::computeNpredArray}
which sums the source model over all pixels:
\begin{equation}
S^{(m)}(E_k) = \sum_i s^{i,m}(E_k)
\end{equation}
where
$s^{i,m}(E_k)$ are the {\tt srcmap} pixels of model $m$ at energy $E_k$.

The $s^{i,m}$ are computed in the {\tt SourceMap::SourceMap} constructor.
For a point source, the model is computed using
\begin{equation}
s^{i,m}(E_k) = P(\delta, E_k) \, A(E_k) \, \Omega_i \, C(E_k)
\end{equation}
where
$P(\delta, E_k)$ is the point spread function for a given source position and
a sky pixel at an angular separation of $\delta$ from that position,
$A(E_k)$ is the exposure,
$\Omega_i$ is the solid angle of the pixel $i$, and
$C(E_k)$ is an (optional) correction factor (see {\tt applyPsfCorrections}).

The point spread function for a source at a given location observed in a given
observation is implemented by {\tt MeanPsf.cxx}.
In the initialization step of {\tt MeanPsf} two arrays are set up:
{\tt m\_exposure} (one value for each of the $M$ energy boundaries) and
{\tt m\_psfValues} (matrix of $N \times M$ values, where $N=200$ and $M$ is
the number of energy boundaries).
The {\tt m\_exposure} array stores
\begin{equation}
A(E_k) = \sum_r v(\vec{p}, A(E_k, r), E_k)
\end{equation}
where 
$\vec{p}$ is the location of the point source, and
$r$ is running over the response matrices (e.g. {\tt FRONT} and {\tt BACK}).
The {\tt m\_psfValues} array stores
\begin{equation}
P(\alpha, E_k) = \frac{1}{A(E_k)} \sum_r v(\vec{p}, P(\alpha, E_k, r) \times A(E_k, r), E_k)
\end{equation}
where
$\alpha$ is the logarithmically spaced angular separation from the point
source direction, an array running from $10^{-4}$ to $70$ in $200$ steps.

The {\tt ExposureCube::value} method returns
\begin{equation}
v(\vec{p}, F, E) = \sum_{ij} f^{(1)}(E) X_{ij}^{(1)}(\vec{p}) F_{ij} + f^{(2)}(E) X_{ij}^{(2)}(\vec{p}) F_{ij}
\end{equation}
where 
$f^{(1)}(E)$ and $f^{(2)}(E)$ are the energy dependent lifetime factors,
$X_{ij}^{(1)}$ and $X_{ij}^{(2)}$ are the {\tt EXPOSURE} and the 
{\tt WEIGHTED\_EXPOSURE} in the lifetime cube files,
$F_{ij}$ is an abitrary function,
$i$ and $j$ are indices that run over $\cos \theta$ and $\phi$,
$\vec{p}$ is the incident photon direction, and
$E$ is the energy.
If no trigger rate and energy-dependent efficiency corrections are
applied then $f_1=1$ and $f_2=0$, i.e. the weighting is only done
over the {\tt EXPOSURE} cube:
\begin{equation}
v(\vec{p}, F) = \sum_{ij} f^{(1)}(E) X_{ij}^{(1)}(\vec{p}) F_{ij} .
\end{equation}

The binning of $X_{ij}$ is implemented by the {\tt healpix::CosineBinner} class which
provides a floating point vector as a set of bins in $\cos \theta$ and $\phi$, the
$\phi$ dependence being optional.
Two $\cos \theta$ binning methods are available:
$\cos \theta_i = 1 - \left( \frac{i+0.5}{N} \right)^2 (1 - \cos \theta_{\rm min})$
(the default) and
$\cos \theta_i = 1 - \frac{i+0.5}{N} (1 - \cos \theta_{\rm min})$.
The $\phi$ binning is given by
$\phi_j = \frac{\pi}{4} (j+0.5) / M$.
Typically, $N=40$, $M=15$ and $\cos \theta_{\rm min} = 0$.

The function $F_{ij}$ needs to implement an operator $F(\cos \theta)$ and the
method $F.integral(\cos \theta, \phi)$.
Typical examples of $F$ are the effective area $A_{\rm eff}$ and the PSF $PSF$.





%% TODO

For a diffuse source, the constructor makes use of
{\tt WcsMap::diffuseMap} to allocate a diffuse map and
{\tt WcsMap::convolvedMap} to convolve the map.
The model pixels are then multiplied by the solid angle of the pixels
(method {\tt Pixel::solidAngle}).


%%%%%%%%%%%%%%%%%%%%%%%%%%%%%%%%%%%%%%%%%%%%%%%%%%%
\subsubsection{Solid angle}

The solid angle $\Omega$ of a pixel is computed by the {\tt WcsMap::solidAngle} method
using
\begin{equation}
\Omega = | \Delta l \Delta b |
\end{equation}
where
\begin{eqnarray}
\Delta l & = & \arccos \left( \sin b_{\rm left} \sin b_{\rm right} + \cos b_{\rm left} \cos b_{\rm right}
                         \cos ( l_{\rm left} - l_{\rm right} \right) \\
\Delta b & = & b_{\rm top} - b_{\rm bottom}
\end{eqnarray}
(both in radians) and
$(l,b)_{\rm left} = (i_l-0.5,i_b)$,
$(l,b)_{\rm right} = (i_l+0.5,i_b)$,
$(l,b)_{\rm top} = (i_l,i_b-0.5)$, and
$(l,b)_{\rm bottom} = (i_l,i_b+0.5)$,
with $(i_l,i_b)$ being the pixel index in longitude and latitude, respectively.

\end{document}
